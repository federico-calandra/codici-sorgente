\documentclass[10pt,a4paper]{article}

%% PACCHETTI AGGIUNTIVI
\usepackage{amssymb,amsmath,amsthm,amsfonts}
%\usepackage{bm}
\usepackage{calc}
%\usepackage[inline]{enumitem}
%\usepackage{ifthen}
%\usepackage[utf8]{inputenc}
\usepackage[portrait]{geometry}
%\usepackage{graphicx}
%\usepackage[colorlinks=true,citecolor=blue,linkcolor=blue]{hyperref}
\usepackage{mathrsfs}
%\usepackage{multicol,multirow}
%\usepackage{subcaption}
%\usepackage{tabularx}
%\usepackage[absolute]{textpos}
%\usepackage{titlesec}
%\usepackage{wrapfig}
%\usepackage{xfrac}


%% GEOMETRIA
\geometry{top=1cm,bottom=1cm,left=.7cm,right=.7cm}

%%	STILE
\pagestyle{empty}
%\raggedright

%%	HEADINGS
%%		Numerazione
\setcounter{secnumdepth}{0}
\setlength{\parindent}{3pt}
\setlength{\parskip}{0pt plus 0.5ex}

%%		Formattazione capitoli
%\titleformat{\chapter}[hang]{\normalfont\huge\bfseries}{\thechapter\quad}{0cm}{}  	% hang, block, display, runin

%%		Formattazione headings \titlesec {spazio-sx}{spazio-prima}{spazio-dopo}
%\titlespacing*{\section}{-1.5ex}{1ex}{0ex}
%\titlespacing*{\subsection}{0ex}{.3ex}{0ex}
%\titlespacing*{\subsubsection}{0ex}{0ex}{0ex}

%%		Altra formattazione
%\makeatletter
%\renewcommand{\section}{\@startsection{section}{1}{-3mm}{2ex}{.1ex}{\normalfont\large\bfseries}}
%\renewcommand{\subsection}{\@startsection{subsection}{2}{0mm}{.5ex}{.1ex}{\normalfont\normalsize\bfseries}}
%\renewcommand{\subsubsection}{\@startsection{subsubsection}{3}{1mm}{.1ex}{.1ex}{\normalfont\small\bfseries}}
%\makeatother

%%	DEFINIZIONE COMANDI
\newcommand{\de}{\mathrm d}
\newcommand{\fracd}[2]{\frac{\de #1}{\de #2}}
\newcommand{\fracp}[2]{\frac{\partial #1}{\partial #2}}
\newcommand{\fracpq}[2]{\frac{\partial^2 #1}{{\partial #2}^2}}
\newcommand{\fracpp}[3]{\frac{\partial^2 #1}{\partial #2 \partial #3}}
\newcommand{\grad}[1]{\text{grad}\,#1}
\newcommand{\dive}[1]{\text{div}\,#1}
\newcommand{\rot}[1]{\text{rot}\,#1}
\newcommand{\vers}{\mathop{\text{vers}}}
\newcommand{\itemm}[1]{\indent - #1\\}
\newcommand{\tr}[1]{\text{tr}\,#1}
\newcommand{\sym}[1]{\text{sym}\,#1}
\newcommand{\skw}[1]{\text{skw}\,#1}
\newcommand{\sz}[1]{\scriptsize #1\normalsize}
\newcommand{\mach}{\text{Ma}}


\begin{document}
	%\tableofcontents\newpage
	\section{}
Considero $f=f(x,y,\dot y)$ definita su una traiettoria $y=y(x)$ con $x\in[x_1,x_2]$. Definisco l'integrale $J=\int_{x_1}^{x_2}f(x,y,\dot y)\de x$. Il problema è quello di trovare un cammino $y(x)$ lungo il quale $J$ è stazionario. Considero la famiglia ad un parametro di traiettorie $y(x,\alpha)=y_0(x)+\alpha\,\eta(x)$, con le condizioni $\eta(x_1)=\eta(x_2)=0$ per $\eta$ arbitraria ma sufficientemente regolare. Se $y_0$ è il cammino "stazionario" allora $y$ è un cammino "variato" rispetto a $y_0$ di una certa quantità $\delta y$ dipendente da $\eta$ e $\alpha$. Con questo artifizio, si ha la dipendenza $J=J(\alpha)$, e la condizione per la stazionarietà di $J$ diventa $\fracd{J}{\alpha}|_{\alpha=0}=0$. Sviluppando la derivata nell'integrale si arriva a $\fracd{J}{\alpha}=\int_{x_1}^{x_2}\fracp{f}{y}\fracp{y}{\alpha}\de x + \int_{x_1}^{x_2}\fracp{f}{\dot y}\fracp{\dot y}{\alpha}\de x$. Il secondo integrale ($\fracp{\dot y}{\alpha} = \fracpp{y}{\alpha}{x}$) lo semplifico per parti, considerando che per le condizioni imposte $y(x_1,\alpha)$ e $y(x_2,\alpha)$ sono costanti in $\alpha$, dunque $\fracd{J}{\alpha}=\int_{x_1}^{x_2}(\fracp{f}{y}-\fracd{}{x}\fracp{f}{\dot y}) {\fracp{y}{\alpha}}\de x$. Per la stazionarietà deve essere quindi $\int_{x_1}^{x_2}[\fracp{f}{y}-\fracd{}{x}(\fracp{f}{\dot y})]{\fracp{y}{\alpha}}|_{\alpha=0} = 0$, e data l'arbitrarietà di $\eta$ posso usare il lemma fondamentale del calcolo delle variazioni ottenendo $\fracp{f}{y}-\fracd{}{x}(\fracp{f}{\dot y})=0$. Questa è l'equazione di Eulero-Lagrange.\\
Posso riscrivere questa equazione in modo più significativo: se mi discosto da $y_0$ di una quantità $\delta y=\fracp{y}{\alpha}|_{\alpha=0}\de\alpha$ allora $J$ subirà una variazione $\delta J=\fracp{J}{\alpha}|_{\alpha=0}\de\alpha$. Riscrivendo l'equazione sopra trovo $\delta J=\int_{x_1}^{x_2}(\fracp{f}{y}-\fracd{}{x}\fracp{f}{\dot y})\delta y\de x$. Necessario quindi che $y$ siano indipendenti. Con le sostituzioni $x \to t, y \to q, \dot y \to \dot q, f \to L$ si trovano le equazioni di Lagrange per sistemi olonomi.\\
Si può generalizzare quanto sopra al caso in cui $f = f(x, y_1, \ldots, y_n, \dot y_1,\ldots,\dot y_n)$. Per ogni variabile $y_j$ considero la famiglia di curve $y_j(x,\alpha) = y_{0j}(x) + \alpha\eta_j(x)$. Questa volta sarà $\fracd {J}{\alpha} = \int_{x_1}^{x_2} (\sum_j \fracp{f}{y_j} \fracp{y_j}{\alpha} + \fracp{f}{\dot y_j} \fracpp{y_j}{x}{\alpha}) \de x$ ma i calcoli sono come sopra. Ogni termine della seconda somma lo sviluppo per parti e alla fine ottengo $\fracd {J}{\alpha} = \int_{x_1}^{x_2} \sum_j (\fracp{f}{y_j} - \fracd{}{x} \fracp{f}{\dot y_j}) \fracp{y_j}{x}\de x$. Per la stazionarietà $\fracp{J}{\alpha} = 0 \Rightarrow \fracp{f}{y_j} - \fracd{}{x} \fracp{f}{\dot y_j} = 0$ per ogni $j$. Le soluzioni delle equazioni rappresentano i cammini $y_j = y_j(x)$ lungo i quali $J$ è costante.
	
	\section{}
Parto dal principio di D'Alambert con l'ipotesi di vincoli ideali $\sum_{ij} \mathbf F^{att}_i \cdot \fracp{\mathbf r_i}{q_j} \delta q_j - \sum_{ij}  m_i \ddot{\mathbf r}_i \cdot \fracp{\mathbf r_i}{q_j} \delta q_j = 0$. Uso l'identità della derivata del prodotto $\fracd{}{t}\{m_i \dot{\mathbf r}_i \cdot \fracp{\mathbf r_i}{q_j}\}$ per espandere il secondo termine $\sum_{ij} m_i \ddot{\mathbf r}_i \cdot \fracp{\mathbf r_i}{q_j} \dot q_j = \sum_{ij} (\fracd{}{t}\{m_i \dot{\mathbf r}_i \cdot \fracp{\mathbf r_i}{q_j}\} - m_i \dot{\mathbf r}_i \cdot \fracd{}{t}\fracp{\mathbf r_i}{q_j})$. L'obiettivo è tradurre $\dot{\mathbf r}_i$, $
\fracp{\mathbf r_i}{q_j}$ e $\fracd{}{t}\fracp{\mathbf r_i}{q_j}$ in quantità con $\mathbf v$ al posto di $\mathbf r$. Per la prima è banalmente $\dot{\mathbf r}_i = \mathbf v_i$. Per la seconda, dall'identità $\mathbf v = \fracp{\mathbf r}{q_1} \dot q_1 + \fracp{\mathbf r}{q_2} \dot q_2 + \dots + \fracp{\mathbf r}{t}$ segue banalmente che $\forall j$ vale $\fracp{\mathbf v}{\dot q_j} = \fracp{\mathbf r}{q_j}$. Per la terza, (supposto che sia lecito farlo) inverto l'ordine di derivazione $\fracd{}{t}\fracp{\mathbf r_i}{q_j} = \fracp{}{q_j} \fracd{\mathbf r_i}{t} = \fracp{\mathbf v_i}{q_j}$. Così ho ottenuto la riformulazione del principio di D'Alambert [FINIRE]
 
	
\end{document}